%-----------------------------------------------------------------------------------------------
\section{Introduction}

    The subject of the second law  of  thermodynamics  figures  amidst  the  most  philosophical
    teaching topics in mechanical engineering. Many studies have been produced around the  theme
    of pedagogical improvements in presenting the second law of thermodynamics, under a  variety
    of    strategies~\cite{1995-MoukalledF+NuwayhidRY-IJMEE,     1996-KaufmanR+SheldonE-AmJPhys,
    1997-BejanA-IJMEE,              1997-DunbarWR+LiorN-IJMEE,               2011-LewinsJ-IJMEE,
    2015-BeaubouefSr-IntJMechEngEduc}.

    One of the key concepts covered  in  textbook  presentations  of  the  second  law  are  the
    so-called          Carnot          principles~\cite{2013-CengelYA+BolesMA-AMGH}           or
    corolaries~\cite{2002-MoranMJ+ShapiroHN-LTC}, or  claim~\cite[p.~88]{2006-BejanA-Wiley},  or
    even deductions than make up  one  principle~\cite{1986-JonesJB+HawkinsGA-Wiley}---different
    sources attribute different names and list a varying number of what Carnot has mostly called
    \emph{propositions}~\cite{1897-ThurstonRH-Wiley}, without labeling them, except for the ones
    that he named ``fundamental'' and ``general''.

    Carnot's fundamental proposition is a statement of a necessary condition for heat engines to
    attain ``maximum'' work production from heat~\cite[p.~56]{1897-ThurstonRH-Wiley}:

    \begin{quote}
        \it
        ``[...] that in the bodies employed to realize the motive  power
        of heat there should not occur any change of  temperature  which
        may not be due to a change of volume.''
    \end{quote}

    Recall his work appears many  years  before  the  second  law  of  thermodynamics  had  been
    established, and the properties we now use, such as entropy, had been  invented.  In  modern
    terms, Carnot's fundamental proposition can be  paraphrased  as  ``the  temperature  of  the
    working fluid can only change during isentropic changes in  volume'',  recalling  that  both
    friction and heat transfer irreversibilities would cause the fluid to experience temperature
    changes that could not be explained by adiabatic and reversible change in volume.

    Carnot's general proposition, on the other hand, is the generalization  of  his  fundamental
    proposition, and states that~\cite[p.~68]{1897-ThurstonRH-Wiley}:

    \begin{quote}
        \it
        ``The motive power of heat is independent of the agents employed
        to realize it; its quantity is fixed solely by the  temperatures
        of the bodies between, which is effected, finally, the  transfer
        of the caloric.''
    \end{quote}

    Meaning, in current thermodynamic terms, that the maximum work producing potential  of  heat
    is independent of the working fluid substance employed in its production,  being  determined
    only by the temperatures\footnote{Note the plural which embeds what would later on  make  up
    the Kelvin-Planck statement of the second law.} in which heat is  being  transfered.  Modern
    texts make it explicit that such  work  producing  potential  is  also  independent  on  the
    engine's internal details~\cite{2013-CengelYA+BolesMA-AMGH}, or, equivalently,  sequence  of
    logical~\cite{2020-NaaktgeborenC-engrXiv} processes~\cite{2002-MoranMJ+ShapiroHN-LTC} of the
    reversible machines.

    Carnot's general proposition is typically presented to mechanical  engineering  students  in
    the context of the second law of thermodynamics in the form of a  statement  followed  by  a
    proof   by    contradiction~\cite{2013-CengelYA+BolesMA-AMGH,    2002-MoranMJ+ShapiroHN-LTC,
    1986-JonesJB+HawkinsGA-Wiley},  using  second  law  concepts  and  abstractions,  i.e.,   by
    proposing a violation of the general proposition and showing in the conceptual realm that it
    leads to a violation of a statement of the second law of thermodynamics---a method  that  is
    both sufficient and general from a mathematical viewpoint.

    Despite the mathematical fitness of this approach, it lacks concrete examples. In one  hand,
    one may argue that lack of concrete examples may  pose  unnecessary  extra  difficulties  on
    learners; on the other hand, the proposition's phrasing seems to beg for concrete  examples,
    as it mentions concrete things such as working fluids and engine internal details.

    Keen and attentive learners may correctly wonder that both heat intake and net  work  output
    done by reversible models of different real engines may be written as complicated  functions
    of the engines' internal details, working fluid properties, and reservoir temperatures---and
    wonder how all these things may fit together under  the  restrictions  of  Carnot's  general
    proposition.

    This work focuses in documenting one such analysis, and thus becoming a reference  that  can
    be  presented  opportunely---not  in  place  of,  but   alongside   the   usual   proof   by
    contradiction---to inquiring  learners,  or  whenever  an  instructor  see  to  be  fitting.
    Moreover, it is hoped that the concrete aspects of Carnot's general proposition are met,  at
    least in the few instances provided herein.

    Well-known engineering textbook reversible cycles include (i)~the Carnot, (ii)~the Stirling,
    and (iii)~the Ericson  ones~\cite{2013-CengelYA+BolesMA-AMGH}.  The  Stirling  cycle  is  of
    peculiar interest for the task due to (a)~the present time interest  displayed  towards  it;
    (b)~the fact that many reversible models  are  available  in  the  literature;  and  (c)~the
    existence of three different such engine configuration types: the $\alpha$-,  $\beta$-,  and
    $\gamma$-ones---so that variations on engine internal details and operating  parameters  and
    on working  fluid  can  be  performed  in  the  context  of  illustrating  Carnot's  general
    proposition.

    Historically,   reversible    Stirling    engine    cycles    have    been    proposed    by
    Schmidt~\cite{1871-SchmidtG-ZeitVerDeutschIng},
    Finkelstein~\cite{1960-FinkelsteinT-SAEIntl},   Walker~\cite{1962-WalkerG-JMechEngSci}   and
    Kirkley~\cite{1962-KirkleyDW-JMechEngSci}. The focus of early works has  been  investigating
    optimal engine construction and operating parameters.

    More recently, Cheng and Yang~\cite{2012-ChengCH+YangHS-ApEnergy}  developed  a  reversible,
    dimensionless, and parametric Stirling engine model that accounts not only for working fluid
    and for engine operating parameter variations, but also for engine configuration alteration,
    i.e., among the $\alpha$-, the $\beta$-, and  the  $\gamma$-types.  Moreover,  they  provide
    exact analytical expressions for net work output and for heat  inlet  that  are  written  in
    terms of the various model parameters and engine internal configurations.

    Following  reasons  (b)  and~(c)   given   above,   the   existing   work   of   Cheng   and
    Yang~\cite{2012-ChengCH+YangHS-ApEnergy}  is  deemed  ideally  suited  for  the  purpose  of
    illustrating Carnot's general proposition ``in action'' based  on  a  concrete  example---of
    reversible Stirling engines of various types. Therefore, this work expands knowledge by  the
    \emph{connections that are made} and \emph{insights than can be provoked} in association  to
    Carnot's general proposition, and by the \emph{production} and \emph{documentation} of  such
    illustration, rather than developing another Stirling engine model or even expanding  beyond
    Cheng and Yang's \emph{model}.

%-----------------------------------------------------------------------------------------------

