%-----------------------------------------------------------------------------------------------
\makeatletter
\immediate\write18{datelog > \jobname.info}
\makeatother
%-----------------------------------------------------------------------------------------------
% Journal information
\JournalInfo{engrXiv}
\Archive{\input{\jobname.info}revision}
% Article title
\PaperTitle{On Illustrating Carnot's General Proposition by Means of Reversible Stirling Engines}
\Authors{%
    C.~Naaktgeboren\textsuperscript{1$\star$},
    K.~A.~É.~Beck\textsuperscript{2},
    J-M.~S.~Lafay\textsuperscript{3}
}
\affiliation{%
    \textsuperscript{1}%
    \textit{%
        Adjunct Professor.
        Universidade Tecnológica Federal do Paraná -- UTFPR, Câmpus Guarapuava.
        Grupo de Pesquisa em Ciências Térmicas.
}}
\affiliation{%
    \textsuperscript{2}%
    \textit{%
        Graduate Student.
        Universidade Tecnológica Federal do Paraná -- UTFPR, Câmpus Pato Branco.
        Programa de Pós-Graduação em Eng.~Elétrica -- PPGEE.
}}
\affiliation{%
    \textsuperscript{3}%
    \textit{%
        Assistant Professor.
        Universidade Tecnológica Federal do Paraná -- UTFPR, Câmpus Pato Branco.
        Programa de Pós-Graduação em Eng.~Elétrica -- PPGEE.
}}
\affiliation{%
    \textsuperscript{$\star$}%
    \textbf{Corresponding  author}: NaaktgeborenC$\cdot$PhD@gmail$\cdot$com
}
\Keywords{%
    Carnot principles ---
    Carnot general proposition ---
    Second law of Thermodynamics ---
    Reversible Stirling engine models ---
    Thermal efficiency
}
\newcommand{\keywordname}{Keywords}
\Highlights{%
    Revisits Carnot's `fundamental' and `general' propositions ---  Identifies  a  link  between
    Carnot's general proposition and the later Kelvin-Planck statement  of  the  second  law  of
    Thermodynamics --- Draws insights and points out connections between  published  models  and
    thermodynamics theory.
}
\newcommand{\highlightname}{Highlights}
%-----------------------------------------------------------------------------------------------
\Abstract{%
    Carnot's general proposition, also referred to as one of Carnot's  principles,  states  that
    the  work  producing  potential  of  heat---harvested  by   reversible   heat   engines---is
    \emph{independent} on the working fluid  and  on  engine  internal  details,  being  only  a
    function of the temperatures of the reservoirs with which the engine  exchanges  heat.  This
    concept,  usually  presented  to  ME  students  in  the  context  of  the  second   law   of
    thermodynamics,  is  usually  proven  by  contradiction,  using  second  law  concepts   and
    abstractions, without concrete examples, even though Carnot's proposition mentions  concrete
    things such as working fluids and engine internal details. This work  proposes  to  document
    the usage of reversible Stirling engine models that take the engine  arrangement  and  fluid
    properties into account towards illustrating the validity of Carnot's  general  proposition.
}
%-----------------------------------------------------------------------------------------------
