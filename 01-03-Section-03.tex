%-----------------------------------------------------------------------------------------------
\section{Illustration of Carnot's General Proposition}

    Solutions to the integrals of Eqs.~(\ref{eq:W}) and (\ref{eq:Q})  are  given  by  Cheng  and
    Yang~\cite{2012-ChengCH+YangHS-ApEnergy}, which applied the residual theorem. The  resulting
    expressions are:
    %
    \begin{align}
        \label{eq:Ws}
        \bar{W} &= \frac{2\pi\tau\kappa(1 - \tau)\sin\alpha}{a^2 + b^2}\cdot\Lambda,\\
        \label{eq:Qs}
        \bar{Q}_{in} &= \frac{2\pi\tau\kappa\sin\alpha}{a^2 + b^2}\cdot\Lambda,
    \end{align}
    %
    \noindent with
    %
    \begin{equation}
        \label{eq:Lamda}
        \Lambda = \left(
                \frac{\beta - \sqrt{\beta^2 - (a^2 + b^2)}}{\sqrt{\beta^2 - (a^2 + b^2)}}
            \right),
    \end{equation}
    %
    \noindent where $a$, $b$, and $\beta$ are configuration dependent expressions.

    Although the  cycle  net  work,  $\bar{W}$,  and  inlet  heat,  $\bar{Q}_{in}$,  are  indeed
    \emph{complicated} functions of the engine's configuration, internal  details,  and  of  the
    reservoir temperatures, as shown by Eqs.~(\ref{eq:Ws}) and (\ref{eq:Qs}), one  can  realize,
    by mere inspection on these same equations, that the thermal  efficiencies,  $\eta_t  \equiv
    \bar{W} / \bar{Q}_{in}$, of all these reversible engines  (i)~are  indeed  \emph{the  same};
    (ii)~are \emph{equal to any reversible engine's} thermal efficiency,  $\eta_{t,rev}  =  1  -
    \tau$, operating between the same temperature reservoirs; and (iii)~are indeed a function of
    \emph{only} the reservoirs' temperatures.

    Carnot's general proposition is thus \emph{validated} by reversible models of real  Stirling
    engine configurations.

    For completeness, the following are the configuration dependent expressions of $a$, $b$, and
    $\beta$:
    %
    \begin{align}
        \label{eq:aa}
        a^{\alpha}  &= \tau\sin\alpha,\\
        \label{eq:ab}
        a^{\beta}   &= -(1 - \tau)\sin\alpha,\\
        \label{eq:ag}
        a^{\gamma}  &= -(1 - \tau)\sin\alpha;\\
        \label{eq:ba}
        b^{\alpha}  &= \kappa + \tau\cos\alpha,\\
        \label{eq:bb}
        b^{\beta}   &= \kappa - (1 - \tau)\cos\alpha,\\
        \label{eq:bg}
        b^{\gamma}  &= \kappa - (1 - \tau)\cos\alpha;\\
        \label{eq:betaa}
        \beta^{\alpha}  &= 4\tau\chi/(1 + \tau) + \tau + \kappa,\\
        \label{eq:betab}
        \beta^{\beta}   &= 4\tau\chi/(1 + \tau) + \tau \nonumber\\
                        &+ \sqrt{1 - 2\kappa\cos\alpha + \kappa^2},\\
        \label{eq:betag}
        \beta^{\gamma}  &= 4\tau\chi/(1 + \tau) + 1 + \tau + \kappa.
    \end{align}

%-----------------------------------------------------------------------------------------------

