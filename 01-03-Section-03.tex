%-----------------------------------------------------------------------------------------------
\section{Illustration of Carnot's General Proposition}

    Solutions to the integrals of Eqs.~(\ref{eq:W}) and (\ref{eq:Q})  are  given  by  Cheng  and
    Yang~\cite{2012-ChengCH+YangHS-ApEnergy}, which applied the residual theorem. The  resulting
    expressions are:
    %
    \begin{align}
        \label{eq:Ws}
        \bar{W} &= \frac{2\pi\tau\kappa(1 - \tau)\sin\alpha}{a^2 + b^2}\cdot\Lambda,\\
        \label{eq:Qs}
        \bar{Q}_{in} &= \frac{2\pi\tau\kappa\sin\alpha}{a^2 + b^2}\cdot\Lambda,
    \end{align}
    %
    \noindent with
    %
    \begin{equation}
        \label{eq:Lamda}
        \Lambda = \left(
                \frac{\beta - \sqrt{\beta^2 - (a^2 + b^2)}}{\sqrt{\beta^2 - (a^2 + b^2)}}
            \right),
    \end{equation}
    %
    \noindent where $a$, $b$, and $\beta$ are configuration dependent expressions, given as:
    %
    \begin{align}
        \label{eq:aa}
        a^{\alpha}  &= \tau\sin\alpha,\\
        \label{eq:ab}
        a^{\beta}   &= -(1 - \tau)\sin\alpha,\\
        \label{eq:ag}
        a^{\gamma}  &= -(1 - \tau)\sin\alpha;\\
        \label{eq:ba}
        b^{\alpha}  &= \kappa + \tau\cos\alpha,\\
        \label{eq:bb}
        b^{\beta}   &= \kappa - (1 - \tau)\cos\alpha,\\
        \label{eq:bg}
        b^{\gamma}  &= \kappa - (1 - \tau)\cos\alpha;\\
        \label{eq:betaa}
        \beta^{\alpha}  &= 4\tau\chi/(1 + \tau) + \tau + \kappa,\\
        \label{eq:betab}
        \beta^{\beta}   &= 4\tau\chi/(1 + \tau) + \tau \nonumber\\
                        &+ \sqrt{1 - 2\kappa\cos\alpha + \kappa^2},\\
        \label{eq:betag}
        \beta^{\gamma}  &= 4\tau\chi/(1 + \tau) + 1 + \tau + \kappa.
    \end{align}

    Although the  cycle  net  work,  $\bar{W}$,  and  inlet  heat,  $\bar{Q}_{in}$,  are  indeed
    \emph{complicated} functions of the engine's configuration, internal  details,  and  of  the
    reservoir temperatures, as shown by Eqs.~(\ref{eq:Ws}) and (\ref{eq:Qs}), one  can  realize,
    by mere inspection on these same equations, that the thermal  efficiencies,  $\eta_t  \equiv
    \bar{W} / \bar{Q}_{in}$, of all these reversible engines:
    %
    \begin{enumerate}
        \item Are indeed \emph{the same}: $\eta^{\alpha}_t = \eta^{\beta}_t = \eta^{\gamma}_t$;
        \item Are \emph{equal to any reversible engine's} thermal efficiency, $\eta_{t,rev} =  1
            - \tau$, operating between the same temperature reservoirs; and
        \item  Are  indeed   a   function   of   \emph{only}   the   reservoirs'   temperatures:
            $\eta_t\!:\!\eta_t(\tau)$.
    \end{enumerate}

    Therefore, the reversible $\alpha$-, $\beta$-, and $\gamma$-types Stirling engine models  of
    Cheng and Yang~\cite{2012-ChengCH+YangHS-ApEnergy}, are shown  to  follow  Carnot's  general
    proposition, according to the second law of thermodynamics.

    The worked out illustration serves thus as an instance of defined  reversible  engines  with
    different internal details and  working  fluids  of  possibly  different  properties,  whose
    thermal efficiencies are restricted by Carnot's general proposition, despite the  fact  that
    their net work and inlet heat are different and complicated functions of the various working
    parameters and engine configuration.

%-----------------------------------------------------------------------------------------------

